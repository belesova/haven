%%%%%%%%%%%%%%%%%%%%%%%%%%%%%%%%%%%%%%%%%%%%%%%%%%%%%%%%%%%%%%%%%%%%%%%%%%%%%%%%
\documentclass[draft]{llncs}
%%%%%%%%%%%%%%%%%%%%%%%%%%%%%%%%%%%%%%%%%%%%%%%%%%%%%%%%%%%%%%%%%%%%%%%%%%%%%%%% 

\usepackage{url}
\usepackage[usenames]{color}
\usepackage{pslatex}
\usepackage{epsfig}
\usepackage{paralist}
\usepackage{wrapfig}
%\usepackage[a]{esvect}

\overfullrule=5pt


% \pagestyle{plain}
\pagestyle{empty}

%\pdfpagesattr{/CropBox [92 62 523 728]} % LNCS page made big 

%%%%%%%%%%%%%%%%%%%%%%%%%%%%%%%%%%%%%%%%%%%%%%%%%%%%%%%%%%%%%%%%%%%%%%%%%%%%%%%%

\title{Better-than-Veloce Accelerated Verification
\thanks{This work was supported by the FIT grant FIT-11-S-1 and the research plan 
MSM0021630528.}\vspace*{-0mm}}


\author{Michal Kajan%\inst{1}
\and  Ond\v{r}ej Leng\'{a}l%\inst{1}
\and  Marcela \v{S}imkov\'{a}%\inst{1}
}

\institute{ {FIT, Brno University of Technology, Czech Republic}
%\and {Institution 2}
%\and {Institution 3}
}

%%%%%%%%%%%%%%%%%%%%%%%%%%%%%%%%%%%%%%%%%%%%%%%%%%%%%%%%%%%%%%%%%%%%%%%%%%%%%%%%
\begin{document} 
%%%%%%%%%%%%%%%%%%%%%%%%%%%%%%%%%%%%%%%%%%%%%%%%%%%%%%%%%%%%%%%%%%%%%%%%%%%%%%%%

\maketitle

\vspace*{-0mm}\begin{abstract}Functional verification is a widespread technique 
to check whether a hardware system satisfies a given correctness specification. 
The complexity of modern computer systems is rapidly rising and the 
verification process takes a significant amount of time. It is a challenging 
task to find appropriate acceleration techniques for this process. In this 
paper we present a design of a verification framework that exploits the 
field-programmable gate array (FPGA) technology, while retaining the 
possibility to run verification in the user-friendly debugging environment of a 
simulator. The main advantages (or it is unique in that it) of the framework 
are that it does not require investments in expensive specialized hardware and 
is free to the community for further research and development. According to the 
experiments carried out on a prototype implementation, the achieved 
acceleration is proportional to the number of checked transactions and the 
complexity of the verified system. The maximum acceleration achieved on the set 
of experiments was over 130 times.\end{abstract}
%===============================================================================
\vspace*{-0mm}\section{Introduction}\vspace*{-0mm}
%===============================================================================
Today's highly competitive market of consumer electronics is very sensitive to the 
time it takes to introduce a new product (the so-called time to market). This has 
driven the demand for fast, efficient and cost-effective methods of verification 
of hardware systems. Simulation-based approaches like functional verification 
suffer from the fact that software simulation of inherently parallel hardware is 
extremely slow when compared to the speed of real hardware. The gap between the 
speed of simulation and the speed of real hardware widens with the increasing 
complexity of the hardware design. An effort to increase efficiency and speed of 
simulation or functional verification poses a considerable challenge not only for 
research teams but also for commercial sphere (scope, area, field) (e.g. Mentor 
Graphics' Veloce technology). As representatives of the first mentioned community 
we introduce an open framework that exploits the inherent parallelism of hardware 
designs to accelerate functional verification of these designs by targeting 
special components of the verification environment to the FPGA. This is possible 
because the generic nature of verification methodologies (OVM, UVM) and 
transaction-based communication among their subcomponents make it possible to 
transparently move these subcomponents to a specialized hardware, while 
maintaining the same level of readability to verification engineers.
%===============================================================================
%===============================================================================
\vspace*{-0mm}\section{Related Work}\vspace*{-0mm}
%===============================================================================

%===============================================================================

%===============================================================================
\vspace*{-0mm}\section{Design of a Verification Framework}\vspace*{-0mm}
%===============================================================================

%===============================================================================

%===============================================================================
\vspace*{-0mm}\section{Experimental Results}\vspace*{-0mm}
%===============================================================================

%===============================================================================

%===============================================================================
\vspace*{-0mm}\subsection{Bug hunting}\vspace*{-0mm}
%===============================================================================

%===============================================================================

%===============================================================================
\vspace*{-0mm}\section{Conclusion}\vspace*{-0mm} \label{sec:conclusion}

Here should be the conclusion.

%===============================================================================
\vspace*{-0mm}\subsection{Future Work}\vspace*{-0mm}
%===============================================================================

%===============================================================================

%%%%%%%%%%%%%%%%%%%%%%%%%%%%%%%%%%%%%%%%%%%%%%%%%%%%%%%%%%%%%%%%%%%%%%%%%%%%%%%%

% \bibliographystyle{splncs03}
% \bibliographystyle{plain}

%\bibliography{literature}

{
\begin{thebibliography}{10}

%\fontsize{9pt}{10pt}\selectfont

%\vspace*{-2mm}

\bibitem{parosh-tree-simulations}
P.~A. Abdulla, A.~Bouajjani, L.~Hol\'{\i}k, L.~Kaati, and T.~Vojnar.
\newblock Computing Simulations over Tree Automata: Efficient Techniques for
  Reducing Tree Automata.
\newblock In {\em Proc. of TACAS'08}, LNCS 5148, Springer, 2008.

\bibitem{parosh-tacas10}
P.~A. Abdulla, L.~Hol\'{i}k, Y.-F. Chen, R.~Mayr, and T.~Vojnar.
\newblock When Simulation Meets Antichains (On Checking Language Inclusion of
  Nondeterministic Finite (Tree) Automata).
\newblock In {\em Proc. of TACAS'10}, LNCS 6015, Springer, 2010.

\bibitem{parosh-rtmc}
P.~A. Abdulla, B.~Jonsson, P.~Mahata, and J.~d'Orso.
\newblock Regular Tree Model Checking.
\newblock In {\em Proc. of CAV'02}, LNCS 2404, Springer, 2002.

\bibitem{antichain}
A.~Bouajjani, P.~Habermehl, L.~Hol\'{i}k, T.~Touili, and T.~Vojnar.
\newblock {Antichain-based Universality and Inclusion Testing over
  Nondeterministic Finite Tree Automata}.
\newblock In {\em Proc. of CIAA'08}, LNCS~5148, Springer, 2008.

\bibitem{bouajjani-artmc}
A.~Bouajjani, P.~Habermehl, A.~Rogalewicz, and T.~Vojnar.
\newblock Abstract Regular Tree Model Checking.
\newblock {\em ENTCS}, 149, Elsevier, 2006.

\bibitem{bouajjani-complex}
A.~Bouajjani, P.~Habermehl, A.~Rogalewicz, T.~Vojnar.
\newblock Abstract Regular Tree Model Checking of Complex Dynamic Data
  Structures.
\newblock In {\em Proc. of SAS'06}, LNCS 4134, Springer, 2006.

\bibitem{bourdier-firewalls}
T.~Bourdier.
\newblock Tree Automata-based Semantics of Firewalls.
\newblock In {\em Proc. of SAR-SSI'11}, IEEE, 2011.

\bibitem{bryant86}
R.~E. Bryant.
\newblock Graph-based Algorithms for {Boolean} Function Manipulation.
\newblock {\em IEEE Trans. Computers}, 1986.

\bibitem{mtbdds}
E.M. Clarke, K.L. McMillan, X.~Zhao, M.~Fujita, and J.~Yang.
\newblock Spectral Transforms for Large Boolean Functions with Applications to
  Technology Mapping.
\newblock {\em FMSD}, 10, Springer, 1997.

\bibitem{doyen:antichain}
L.~Doyen and J.~F. Raskin.
\newblock {Antichain Algorithms for Finite Automata}.
\newblock In {\em Proc. of TACAS'10}, LNCS 6015, Springer, 2010.

\bibitem{habermehl-forest}
P.~Habermehl, L.~Hol{\'i}k, A.~Rogalewicz, J.~{\v S}im{\'a}{\v c}ek, and
  T.~Vojnar.
\newblock Forest Automata for Verification of Heap Manipulation.
\newblock In {\em Proc. of CAV'11}, LNCS 6806, Springer, 2011

\bibitem{top-down-TR-11}
L.~Hol\'{i}k, O.~Leng\'{a}l, J.~\v{S}im\'{a}\v{c}ek, and T. Vojnar.
\newblock Efficient Inclusion Checking on Explicit and Semi-Symbolic Tree
  Automata.
\newblock Tech. rep. FIT-TR-2011-04, FIT BUT, Czech Rep., 2011.

\bibitem{hosoya05}
H.~Hosoya, J.~Vouillon, and B.~C. Pierce.
\newblock Regular Expression Types for {XML}.
\newblock {\em ACM Trans. Program. Lang. Syst.}, 27, 2005.

\bibitem{ilie04}
L.~Ilie, G.~Navarro, and S.~Yu.
\newblock On {NFA} Reductions.
\newblock In {\em Proc. of Theory is Forever}, LNCS 3113, Springer, 2004.

\bibitem{monasecrets}
N.~Klarlund, A.~M\o{}ller, and M.~I. Schwartzbach.
\newblock {MONA} Implementation Secrets.
\newblock {\em International Journal of Foundations of Computer Science},
  13(4), 2002.

\bibitem{madhusudan-decidable}
P.~Madhusudan, G.~Parlato, and X.~Qiu.
\newblock Decidable Logics Combining Heap Structures and Data.
\newblock {\em SIGPLAN Not.}, 46, 2011.

\bibitem{cudd}
F.~Somenzi.
\newblock {CUDD: CU Decision Diagram Package Release 2.4.2}, May 2011.

\bibitem{tozawa-xml}
A.~Tozawa and M.~Hagiya.
\newblock XML Schema Containment Checking Based on Semi-implicit Techniques.
\newblock In {\em Proc. of CIAA'03}, LNCS 2759, Springer, 2003.

\bibitem{wulf:antichains}
M.~De Wulf, L.~Doyen, T.~A. Henzinger, J.-F. Raskin.
\newblock {Antichains: A New Algorithm for Checking Universality of Finite
  Automata}.
\newblock In {\em Proc. of CAV'06}, LNCS 4144, Springer, 2006.

\end{thebibliography}
}

%%%%%%%%%%%%%%%%%%%%%%%%%%%%%%%%%%%%%%%%%%%%%%%%%%%%%%%%%%%%%%%%%%%%%%%%%%%%%%%%
\end{document}
%%%%%%%%%%%%%%%%%%%%%%%%%%%%%%%%%%%%%%%%%%%%%%%%%%%%%%%%%%%%%%%%%%%%%%%%%%%%%%%%
