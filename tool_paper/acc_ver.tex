%%%%%%%%%%%%%%%%%%%%%%%%%%%%%%%%%%%%%%%%%%%%%%%%%%%%%%%%%%%%%%%%%%%%%%%%%%%%%%%%
\documentclass[draft]{llncs}
%%%%%%%%%%%%%%%%%%%%%%%%%%%%%%%%%%%%%%%%%%%%%%%%%%%%%%%%%%%%%%%%%%%%%%%%%%%%%%%%


\usepackage{url}
\usepackage[usenames]{color}
\usepackage{pslatex}
\usepackage{epsfig}
\usepackage{paralist}
\usepackage{wrapfig}
%\usepackage[a]{esvect}

%\overfullrule=5pt

\newcommand{\notice}[1]{\textcolor{red}{#1}}

\newcommand{\nat}{{N}}
\newcommand{\bool}{{B}}
\newcommand{\tr}{\rightarrow} % transition ->
\newcommand{\ltr}[1]{\stackrel{#1}{\longrightarrow}} % labelled->
\newcommand{\run}{\Longrightarrow}
\newcommand{\lrun}[1]{\stackrel{#1}{\Longrightarrow}} % labelled->
\newcommand{\ltrlow}[1]{\mathrel{\raisebox{0pt}[-0.10ex][0pt]
  {\ensuremath{\stackrel{#1}{\raisebox{0pt}[0.57ex][0pt]{
  \ensuremath{\longrightarrow}}}}}}} % labelled->
\newcommand{\nset}[1]{\{1,\ldots,#1\}}
\def\asym<#1>{\langle#1\rangle}

\newcommand{\lang}{\mathcal{L}}
\newcommand{\F}{\mathcal{F}}
\newcommand{\A}{\mathcal{A}}
\newcommand{\B}{\mathcal{B}}
\newcommand{\sem}[1]{\[ #1 \]}
\newcommand{\ic}[3]{#2 \rightarrow_#1 #3}
\newcommand{\br}{\mathit{br}}
\newcommand{\subs}[1]{\tilde{#1}}
\newcommand{\powerset}[1]{2^{#1}}

\newcommand{\states}{Q}
\newcommand{\alphabet}{\Sigma}
\newcommand{\trans}{\Delta}
\newcommand{\transbu}{\trans^{bu}}
\newcommand{\transtd}{\trans^{td}}
\newcommand{\final}{F}
\newcommand{\automaton}[1][]{\A_{#1} = (\states_{#1}, \alphabet, \trans_{#1}, \final_{#1})}
\newcommand{\automatonbu}[1][]{\A_{#1} = (\states_{#1}, \alphabet, \transbu_{#1}, \final_{#1})}
\newcommand{\automatontd}[1][]{\A_{#1} = (\states_{#1}, \alphabet, \transtd_{#1}, \final_{#1})}

\newcommand{\issimby}[1][]{\preceq_{#1}}
\newcommand{\isnotsimby}[1][]{\not\issimby[#1]}
\newcommand{\issimbyfe}[1][]{\issimby[#1]^{\forall\exists}}
\newcommand{\isnotsimbyfe}{\isnotsimby^{\forall\exists}}
\newcommand{\simulates}{\succeq}
\newcommand{\simulatesfe}{\simulates^{\exists\forall}}

\newcommand{\rank}[1][]{\#{#1}}
\newcommand{\dom}{\mathit{dom}}

% the such that symbol for set definitions: |
\newcommand{\suchthat}{\;|\;}
% sequences: x_1, ..., x_n
\newcommand{\seq}[2][n]{{#2}_1, \dots, {#2}_{#1}}

\newcommand{\apply}[1][]{\mathit{Apply_{#1}}}
\newcommand{\createmtbdd}{\mathit{CreateMTBDD}}

\newcommand{\op}{\mathit{op}}
\newcommand{\var}{\mathit{var}}
\newcommand{\low}{\mathit{low}}
\newcommand{\high}{\mathit{high}}
\newcommand{\rootnode}{\mathit{root}}
\newcommand{\bddeval}[1]{[#1]}
\newcommand{\bddfromfunc}[1]{\langle#1\rangle}
\newcommand{\zerobddnode}{\mathbf{0}}
\newcommand{\onebddnode}{\mathbf{1}}
\renewcommand{\vec}[1]{\vv{#1}}
\newcommand{\lhss}[1]{\bar{#1}}
\newcommand{\alltrees}{T_{\Sigma}}

\newcommand{\isaccessiblefromthrough}[2]{down_{#2}(#1)}
%\newcommand{\isaccessiblefromthrough}[2]{#1 \triangledown #2}
\newcommand{\isaccessibleupwardsfromthrough}[2]{up_{#2}(#1)}
\newcommand{\choicefunctions}[2]{\mathit{cf}(#1, #2)}
\def\tuplify(#1,#2,#3){\Lambda(#1,#2,#3)}
\newcommand{\maximumtuples}[1]{#1^{\#}}

\newcommand{\algstyle}{\fontsize{8.6}{8.3}\selectfont}

% \pagestyle{plain}
\pagestyle{empty}

%\pdfpagesattr{/CropBox [92 62 523 728]} % LNCS page made big 

%%%%%%%%%%%%%%%%%%%%%%%%%%%%%%%%%%%%%%%%%%%%%%%%%%%%%%%%%%%%%%%%%%%%%%%%%%%%%%%%

\title{Better-than-Veloce Accelerated Verification
  \thanks{This work was supported by FILL IN.}\vspace*{-0mm}}

\author{Put\inst{1} your\inst{2} names\inst{3} here\inst{1,2,3}}

\institute{
  {FIT, Brno University of Technology, Czech Republic}
\and
  {Institution 2}
\and
  {Institution 3}
}

%%%%%%%%%%%%%%%%%%%%%%%%%%%%%%%%%%%%%%%%%%%%%%%%%%%%%%%%%%%%%%%%%%%%%%%%%%%%%%%%
\begin{document} 
%%%%%%%%%%%%%%%%%%%%%%%%%%%%%%%%%%%%%%%%%%%%%%%%%%%%%%%%%%%%%%%%%%%%%%%%%%%%%%%%

\maketitle

\vspace*{-0mm}\begin{abstract}Here should be the abstract.\end{abstract}

%===============================================================================
\vspace*{-0mm}\section{Introduction}\vspace*{-0mm}
%===============================================================================

Here should be introduction


%===============================================================================
\vspace*{-0mm}\section{Conclusion}\vspace*{-0mm} \label{sec:conclusion}
%===============================================================================

Here should be the conclusion.

%%%%%%%%%%%%%%%%%%%%%%%%%%%%%%%%%%%%%%%%%%%%%%%%%%%%%%%%%%%%%%%%%%%%%%%%%%%%%%%%

% \bibliographystyle{splncs03}
% \bibliographystyle{plain}

%\bibliography{literature}

{
\begin{thebibliography}{10}

%\fontsize{9pt}{10pt}\selectfont

%\vspace*{-2mm}

\bibitem{parosh-tree-simulations}
P.~A. Abdulla, A.~Bouajjani, L.~Hol\'{\i}k, L.~Kaati, and T.~Vojnar.
\newblock Computing Simulations over Tree Automata: Efficient Techniques for
  Reducing Tree Automata.
\newblock In {\em Proc. of TACAS'08}, LNCS 5148, Springer, 2008.

\bibitem{parosh-tacas10}
P.~A. Abdulla, L.~Hol\'{i}k, Y.-F. Chen, R.~Mayr, and T.~Vojnar.
\newblock When Simulation Meets Antichains (On Checking Language Inclusion of
  Nondeterministic Finite (Tree) Automata).
\newblock In {\em Proc. of TACAS'10}, LNCS 6015, Springer, 2010.

\bibitem{parosh-rtmc}
P.~A. Abdulla, B.~Jonsson, P.~Mahata, and J.~d'Orso.
\newblock Regular Tree Model Checking.
\newblock In {\em Proc. of CAV'02}, LNCS 2404, Springer, 2002.

\bibitem{antichain}
A.~Bouajjani, P.~Habermehl, L.~Hol\'{i}k, T.~Touili, and T.~Vojnar.
\newblock {Antichain-based Universality and Inclusion Testing over
  Nondeterministic Finite Tree Automata}.
\newblock In {\em Proc. of CIAA'08}, LNCS~5148, Springer, 2008.

\bibitem{bouajjani-artmc}
A.~Bouajjani, P.~Habermehl, A.~Rogalewicz, and T.~Vojnar.
\newblock Abstract Regular Tree Model Checking.
\newblock {\em ENTCS}, 149, Elsevier, 2006.

\bibitem{bouajjani-complex}
A.~Bouajjani, P.~Habermehl, A.~Rogalewicz, T.~Vojnar.
\newblock Abstract Regular Tree Model Checking of Complex Dynamic Data
  Structures.
\newblock In {\em Proc. of SAS'06}, LNCS 4134, Springer, 2006.

\bibitem{bourdier-firewalls}
T.~Bourdier.
\newblock Tree Automata-based Semantics of Firewalls.
\newblock In {\em Proc. of SAR-SSI'11}, IEEE, 2011.

\bibitem{bryant86}
R.~E. Bryant.
\newblock Graph-based Algorithms for {Boolean} Function Manipulation.
\newblock {\em IEEE Trans. Computers}, 1986.

\bibitem{mtbdds}
E.M. Clarke, K.L. McMillan, X.~Zhao, M.~Fujita, and J.~Yang.
\newblock Spectral Transforms for Large Boolean Functions with Applications to
  Technology Mapping.
\newblock {\em FMSD}, 10, Springer, 1997.

\bibitem{doyen:antichain}
L.~Doyen and J.~F. Raskin.
\newblock {Antichain Algorithms for Finite Automata}.
\newblock In {\em Proc. of TACAS'10}, LNCS 6015, Springer, 2010.

\bibitem{habermehl-forest}
P.~Habermehl, L.~Hol{\'i}k, A.~Rogalewicz, J.~{\v S}im{\'a}{\v c}ek, and
  T.~Vojnar.
\newblock Forest Automata for Verification of Heap Manipulation.
\newblock In {\em Proc. of CAV'11}, LNCS 6806, Springer, 2011

\bibitem{top-down-TR-11}
L.~Hol\'{i}k, O.~Leng\'{a}l, J.~\v{S}im\'{a}\v{c}ek, and T. Vojnar.
\newblock Efficient Inclusion Checking on Explicit and Semi-Symbolic Tree
  Automata.
\newblock Tech. rep. FIT-TR-2011-04, FIT BUT, Czech Rep., 2011.

\bibitem{hosoya05}
H.~Hosoya, J.~Vouillon, and B.~C. Pierce.
\newblock Regular Expression Types for {XML}.
\newblock {\em ACM Trans. Program. Lang. Syst.}, 27, 2005.

\bibitem{ilie04}
L.~Ilie, G.~Navarro, and S.~Yu.
\newblock On {NFA} Reductions.
\newblock In {\em Proc. of Theory is Forever}, LNCS 3113, Springer, 2004.

\bibitem{monasecrets}
N.~Klarlund, A.~M\o{}ller, and M.~I. Schwartzbach.
\newblock {MONA} Implementation Secrets.
\newblock {\em International Journal of Foundations of Computer Science},
  13(4), 2002.

\bibitem{madhusudan-decidable}
P.~Madhusudan, G.~Parlato, and X.~Qiu.
\newblock Decidable Logics Combining Heap Structures and Data.
\newblock {\em SIGPLAN Not.}, 46, 2011.

\bibitem{cudd}
F.~Somenzi.
\newblock {CUDD: CU Decision Diagram Package Release 2.4.2}, May 2011.

\bibitem{tozawa-xml}
A.~Tozawa and M.~Hagiya.
\newblock XML Schema Containment Checking Based on Semi-implicit Techniques.
\newblock In {\em Proc. of CIAA'03}, LNCS 2759, Springer, 2003.

\bibitem{wulf:antichains}
M.~De Wulf, L.~Doyen, T.~A. Henzinger, J.-F. Raskin.
\newblock {Antichains: A New Algorithm for Checking Universality of Finite
  Automata}.
\newblock In {\em Proc. of CAV'06}, LNCS 4144, Springer, 2006.

\end{thebibliography}
}

%%%%%%%%%%%%%%%%%%%%%%%%%%%%%%%%%%%%%%%%%%%%%%%%%%%%%%%%%%%%%%%%%%%%%%%%%%%%%%%%
\end{document}
%%%%%%%%%%%%%%%%%%%%%%%%%%%%%%%%%%%%%%%%%%%%%%%%%%%%%%%%%%%%%%%%%%%%%%%%%%%%%%%%
